%%%%%%%%%%%%%%%%%%%%%%%%%%%%%%%%%%%%%%%%%%%%%%%%%%%%%%%%%%%%%%%%%%%%%%%%%%
%                                                                        %
%                        TypeRex OCaml Studio                            %
%                                                                        %
%                           Tiphaine Turpin                              %
%                                                                        %
%  Copyright 2011-2012 INRIA Saclay - Ile-de-France / OCamlPro           %
%  All rights reserved.  This file is distributed under the terms of     %
%  the GNU Public License version 3.0.                                   %
%                                                                        %
%  TypeRex is distributed in the hope that it will be useful,            %
%  but WITHOUT ANY WARRANTY; without even the implied warranty of        %
%  MERCHANTABILITY or FITNESS FOR A PARTICULAR PURPOSE.  See the         %
%  GNU General Public License for more details.                          %
%                                                                        %
%%%%%%%%%%%%%%%%%%%%%%%%%%%%%%%%%%%%%%%%%%%%%%%%%%%%%%%%%%%%%%%%%%%%%%%%%%

%\chapter{Boosting productivity with {\tt ocp-wizard}}

This chapter explains how to enable and use the \typerex\ environment
for editing an OCaml program.

\section{\typerex\ environment setup}

Using the \typerex\ environment for an OCaml project requires two
configuration steps: ensuring the generation of required binary
annotations, and providing a minimal description of the project's
paths. Those steps are detailed in the following.

\subsubsection*{Keeping binary annotations up-to-date}

The binary annotations must be up-to-date for \typerex\ to
function properly, in particular for refactoring, and these
annotations are not updated by \typerex\ itself. This implies that
successive refactoring actions require a recompilation of the impacted
part of the program at each step.

\subsection{Generating .cmt(i) files}

The most simple way of generating binary annotations is to setup your
build process to use the provided \verb!ocp-*! versions of the OCaml
compilers, for example \verb!ocp-ocamlc.opt! instead of
\verb!ocamlc.opt!.  These are wrappers which behave as the original
compilers, but additionally run \verb!ocp-type! on the sources.

In some cases, a more expressive solution is required which
consists in prefixing the compiler commands with
\verb!ocp-wrapper -save-types! with specific arguments (see chapter
``Tools''
%~\ref{chap:tools}\ cross-references don't work
for more details).

Here are examples of how to achieve this depending on your build system.

\subsubsection{make}
Use as compiler a variable defined by
{\verbsize \begin{verbatim}
        OCAMLC=ocp-ocamlc.opt
or
        OCAMLC=ocp-wrapper -save-types [<other options>] ocamlc.opt
\end{verbatim}}
%\noindent and similarly for ocamlopt.

% \subsubsection{ocp-build}
% Add
% {\verbsize \begin{verbatim}
%         ocamlc = ["ocp-wrapper" ; "-save-types" ; "ocamlc.opt"]
% \end{verbatim}}
% \noindent to the ``main'' .ocp file inherited by your whole project.

\subsubsection{ocamlbuild}
Add
{\verbsize \begin{verbatim}
        Options.ocamlc := S [ A "ocp-ocamlc"]
or
        Options.ocamlc :=
          S [ A"ocp-wrapper"; A"-save-types"; ... ; A"ocamlc"]
\end{verbatim}}
\noindent to your \verb!myocamlbuild.ml! file.
Another option is to invoke ocamlbuild with options:
{\verbsize \begin{verbatim}
ocamlbuild -ocamlc ocp-ocamlc.opt -ocamlopt ocp-ocamlopt.opt
\end{verbatim}}
\noindent Finally, don't forget t add \verb!CMT _build! to your \verb!.typerex! file
(see below).

\subsubsection{ocamlfind (without ocamlbuild)}
Add
{\verbsize \begin{verbatim}
        ocamlc = "ocp-ocamlc.opt"
\end{verbatim}}
\noindent or
{\verbsize \begin{verbatim}
        ocamlc(typerex) = "ocp-ocamlc.opt"
\end{verbatim}}
\noindent to your \verb!/etc/findlib.conf! (or \verb!ocamlfind.conf!,
or the file pointed to by \verb!$OCAMLFIND_CONF!).  The first
option tells \verb!ocamlfind! to use \verb!ocp-wrapper! globally ; the
second defines a tool-chain "typerex" which you then specify by calling
{\verbsize \begin{verbatim} ocamlfind -toolchain typerex ocamlc
\end{verbatim}}

\subsubsection{Using a separate build process}
Alternatively, ocp-type provides a file Makefile.ocp-type.template,
which is able to perform the ocp-type compilation automatically for
simple projects.

\subsection{Project configuration: .typerex file}

Most functionalities of \typerex\ rely on some knowledge of the edited
program (source files, libraries\ldots) which should be specified in a
very simple project file at the ``root'' of its source tree with name
\verb!.typerex!.
% in one of the following ways.
%
%\subsubsection{.typerex file}
%This is a very simple configuration file describing a project's
%directories and libraries.
When \typerex\ is invoked on a source file \verb!<file.ml>!, it looks
for file \verb!.typerex! in the directory containing \verb!<file.ml>!,
or its parent directories, back to the file system's root. This file
is read at each command invocation (except syntax coloring and
auto-completion) so modifications are taken into account immediately.

\subsubsection*{Syntax of .typerex files}
The \verb!.typerex! file should specify the set of directories to
search for OCaml source files, and the set of directories to include
in the load path (\emph{i.e.}, libraries). It is also possible to
exclude some source files or whole compilation units, or to force
files to be included whatever their extension (if any).

The syntax of the \verb!.typerex! file is as follows:

{\verbsize\begin{verbatim}
        <project file> := <line>* // file must end with a newline

        <line> := <dirs>        // add all contained source files (non-recursive)
                | I<dirs>       // include <dirs> as external libraries
                                // +<dir> means <stdlib>/<dir>
                | -<files>      // ignore these source files
                | IMPL <files>  // treat these as implementation files
                | INTF <files>  // treat these as interface files
                | CMT <dir>     // search .cmt(i)s here instead of in source dir
                | NOSTDLIB      // do not include the standard library path
                | #<comment>

        <dirs> := white-space-separated list of directories
        <files> := white-space-separated list of files
\end{verbatim}}
\noindent Relative directory names are interpreted with respect to the
directory containing the project file \verb!.typerex!, and the project
directory itself may be denoted by '.', but the shortcuts '$\sim$' and
'$\sim$\verb!user!' are not supported. Note that \verb!-<prefix>! is a
shorthand for \verb!-<prefix>.ml <prefix>.mli ...! See \verb!.typerex! in
the \typerex\ root directory for an example.

\subsubsection*{Meaning of project and library directories}
Lines starting with
\verb!I! indicate that the specified directories are considered as
\emph{library} and not as project's directories. The meaning of this
distinction, which may change in the future, is currently the
following:
\begin{itemize}
\item All source files (.ml, .mli, .mll, .mly) in a project directory
  are considered, whether they have corresponding compiled files
  (.cmi, .cmti, .cmt) or not, while compiled files without sources are
  ignored. This is exactly the opposite for libraries: all .cmi,
  .cmti, .cmt are considered, and uncompiled sources are simply
  ignored.
\item Refactoring and browsing stops at the boundary of libraries, and
  no binding propagation is performed on the implementation of
  libraries (see the documentation for renaming and grep). This saves
  some computation time and is sound unless a library depends on the
  program (but the same question arises when the considered program is
  meant to be a library)
\end{itemize}

\subsubsection*{Pack modules (experimental)}
Pack modules are understood by
\typerex\ if the source directories contain either a file
\begin{itemize}
\item \verb!pack.mlpack! in the ocamlbuild format: a list of module
  names, possibly qualified (using \verb!/!) by a path relative to the
  directory containing the \verb!pack.mlpack! file, or
\item \verb!pack.cmt!, whose contents is a pack module (such as
  generated by \verb!ocp-type -pack!. This option only works if the
  packed modules are in the same directory as the resulting pack,
  which is not the case when compiling with \verb!ocamlbuild!.
\end{itemize}

\subsubsection*{Other options}

{\bf \verb!CMT <dir>! syntax:} It is possible to specify a \verb!CMT!
directory to search for \verb!.cmt(i)! files when they are not found
at the same place as the source files. This is needed if the build
system moves the files around, but then if several modules (in
different directories) have the same name, then outdated cmts won't be
assigned unless there only is one (matching with the source digest to
resolve ambiguity).

{\bf \verb!IMPL <files>!, \verb!INTF <files>!:} Use this to use
\typerex\ on source files with special extension (or none). The
subdirectory containing these files still needs to be specified in the
\verb!.typerex! file. You will also have to enable \typerex\ mode for
those, either manually with \verb!M-x typerex-mode!, or by extending
\verb!auto-mode-alist!  or \verb!interpreter-mode-alist! (see
\verb!emacs.append!).

{\bf \verb!NOSTDLIB!:} Do not implicitely include the standard library
path.  This is required when using \typerex\ on the OCaml compiler.

%An absent project file is allowed and means a project consisting of
%only the current directory, with no library.

%\subsubsection{ocp-build}
%Getting project info from ocp-build is currently disabled.

\subsubsection{Fallback}

If no specific configuration is provided, \typerex\ considers as
program the set of OCaml source files present in the directory
containing the edited source file, with no libraries other than the
OCaml stdlib.

\section{Browsing OCaml code with \typerex}

{\bf Note on browsing commands:} Each cursor motion incurred by a
browsing action (except when clicking on grep results) is undoable
with the standard Emacs shortcut (C-\_).

\subsection{Grep (C-o g / C-o t g)}
(C-o g) display a click-able list (compile minor mode) of the connected
definitions and occurrences of the identifier under the cursor. Invocation
is the same as for renaming. Use (C-o t g) to grep the top-level module
defined by the current file instead of an identifier.

\subsection{Goto-definition (C-o d)}
Places the cursor on the definition of the identifier under the cursor,
opening the appropriate file in the current window if necessary.

\subsection{Cycle-definitions (C-o a)}
Places the cursor on an alternate definition of the identifier declaration
under the cursor, opening the appropriate file in the current window
if necessary. The typical effect is to switch between .ml and .mli
files, but at the right place. This may be used only for top-level
let-bindings (i.e. 'let' and not 'let..in', external statements, type
declarations, exception declarations, and (recursive) module and
module type declarations

\subsection{Comment-definition (C-o c)}
Display a description of the identifier under the cursor, with its lookup
path,
% (unless the cursor is on the definition itself)
and any comments associated with it (in the sense of OCamldoc).
The description is:
\begin{itemize}
\item the type, for a value or field
\item the type declaration, for a type constructor
\item the argument types (or "constant"), for a constructor or exception
\item the module type, for a module or module type.
\end{itemize}
% This command is very similar to the (C-c C-t) command of Tuareg, but
% with the following additional differences:
% \begin{itemize}
% \item It cannot handle partially typed code, but can recover correct
%   information for unmodified parts (even with unsaved buffers)
% \item
% \end{itemize}

\section{Refactoring OCaml code with \typerex}

{\bf Note on reverting and undoing:} For all refactoring actions, the
reverting of modified buffers and the undoing take one of the two
following modes:

\begin{itemize}
\item If the modification is local to the current buffer, then it is
  reverted while keeping its history, and renamed if needed. This
  enables undoing with the standard emacs shortcut (C-\_).
\item If several files are modified, then all relevant buffers are
  reverted and their ``local'' undo-lists are cleared. Instead, the
  multiple-file modification is added to a global undo list and can
  only be undone with ``C-o u''. A call to ``C-o u'' is also pushed
  onto the local undo lists of all modified buffers for convenience,
  so that (C-\_) will also work.
\end{itemize}

\subsection{Multiple-file undo (C-o u)}
Undo the last multiple-file modification. Warning! This discards any
subsequent modification of the affected files (a confirmation is asked
in this case). All buffers editing one of the affected files are
reverted, and their local undo lists are cleared (and then receive a
single new ``global-undo'' item).

\subsection{Renaming (C-o r / C-o t r)}
Rename an identifier through an OCaml program.

\paragraph{(C-o r):} The cursor must be placed on an identifier
definition or reference (for example, a let binding or a pattern).

\paragraph{(C-o t r):} Rename the top-level module defined by the
current file instead of an identifier.\\

\begin{htmlonly}
\smallskip
\end{htmlonly}
Renaming takes care of necessary propagation (e.g., when distinct
values with the same name need to be renamed consistently because this
name appears in a common interface), and capture is detected.

Renaming is implemented for: values, types, modules (non-recursive),
module types, fields, constructors, and exceptions. Aa a convenience,
a partial, unsound renaming of classes and class types is supported,
but will miss all references to the ``secondary'' bindings of a class
or class type, \emph{i.e.}, the closed and open types, and, for a
class, the class type. Type variables, instance variable, methods,
argument labels, and polymorphic variants are not supported.

The replacement is intended to be complete, up to the following known
bugs:
\begin{itemize}
\item labels, e.g. renaming \verb!x! in \verb!let x = .. in f !%
  $\sim$\verb!x! yields \verb!f !$\sim$\verb!y! instead of \verb!f !%
  $\sim$\verb!x:y!, and similarly with \verb!fun !$\sim$\verb!x -> ..!
  \item renaming a type which is in fact a class or a class type, or
    such that its renaming ``propagates'' to one (through module
    constraints and functor applications) will not rename the class or
    class type itself, or its references.
\end{itemize}
Note also the following limitation:
\begin{itemize}
  \item including a module where an element is renamed with an afterwards masked
    name causes a capture error.
\end{itemize}


\subsection{Reference pruning (C-o p)}
Simplify the identifier references (longidents) by removing unnecessary
qualification. This operation ranges on the current buffer.

\subsection{Open elimination (C-o q) (for "qualify")}
Remove (if possible) the open statement under the cursor and qualify
the subsequent references as required. the \verb!let open .. in!
syntax is also supported by open elimination. This operation is
currently slightly conservative, when the same module is opened again
inside one of the items in the elimination scope (sub-modules,
\verb!let open!, and \verb!M.(...)!) but a duplicate open at the same
level will be correctly handled.
% Running \verb!(C-o p)! just after

\section{Syntax coloring}

\typerex\ implements its own version of syntax coloring. It is not
yet fully stable, but already has some new features such as the inline
marking of lexing errors (with help-info) and a smarter treatment of
unterminated strings and comments.

Syntax coloring is not specialized for ocamlyacc/ocamllex files, but
will usually give an acceptable result except for C-style comments.

\section{Auto-completion}

An experimental completion feature is proposed in typerex, currently
only for identifiers (including methods, tags, labels and type
variables). Once enabled, a menu of candidates is triggered when
typing test or with the appropriate key (\verb!<`>! by default) which
also completes the longest common prefix. Other keys allow to select a
candidate and insert it (\verb!<C-n>!, \verb!<C-p>!, and \verb!<\>! by
default), or to cycle between them (with \verb!<TAB>!, see the Auto
Complete Mode user manual).

The candidates computation takes into account the load path which is
configured for the project, the open and include statements and
unqualified identifiers until the current position in the edited file (in a
very approximative and simplistic way) and the module qualification
possibly prefixing the identifier to be completed.

% {\Large Completion is currently broken and unavailable}

% Three completion actions are available:

% \begin{itemize}

% \item Path completion (F12): complete a long identifier (value, record
%   field, module name), interactively, given a (possibly empty) prefix.
%   Empty prefixes are allowed in record (re-)definitions and record
%   patterns, such as:
% \begin{itemize}
%     \item let r $=$ \{f1 = $<$expr$>$ ; .	    (at least one field must be given)
%     \item let r $=$ \{$<$expr$>$ with $<$maybe some fields$>$ .
%     \item in a pattern: \{f1 $=$ $<$patt$>$ .
% \end{itemize}

% \item Pattern expansion (F11): expand a variable or wildcard inside a
%   pattern, according to the inferred sub-pattern's type. The variable
%   under the cursor is only expanded once, possibly introducing new
%   wildcards, and the cursor is positioned on the first created
%   wildcard, if any (allowing further expansion by typing F11 again).

% \item Match cases completion (F3): create a pattern-matching template
%   after a match-with construct. This may be invoked just after:
% \begin{itemize}
%     \item match $<$expr$>$ .
%     \item match $<$expr$>$ with .
%     \item match $<$expr$>$ with $<$match cases$>$ .
% \end{itemize}
%   The result is very similar to pattern expansion, except that for a
%   constructor type (most frequent case), constructors are put on
%   successive lines.
% \end{itemize}


% {\bf Remark:} Completion still needs to be improved in many respects. See
% the test cases for a detailed overview of the current behavior.

\section{\typerex\ assumptions and supported code}

\subsection{Preprocessors}

The browsing commands of \typerex\ support ocamlyacc/ocamllex sources,
and should work with other pre-processors which generate OCaml source
files with appropriate line numbers directives.
%
More precisely, the identifiers in a pre-processed source file which are
actual identifiers of the source (i.e., not generated or transformed during
pre-processing) should be OK to grep or jump from and to, if no
generated code has the same location.

For ocamlyacc and ocamllex files, these "actual" identifiers correspond to
the quoted OCaml code (between braces). Jumping to ocamlyacc entry
points is not supported however, because the generated interface has
no line number directives.
%
Renaming may work in pre-processed or ocamllex/ocamlyacc source files,
but has not been thoroughly tested. Other refactoring commands won't
work on ocamlyacc/ocamllex sources.

The \verb!camlp4! pre-processor (version 3.12.1) is supported, but
only partially because its output is an ast which has insufficient
location information (or a source file but without line numbers
directives).  \verb!ocp-type! (or \verb!ocp-wrapper!) can generate
binary annotations with \verb!camlp4!, but the result of \typerex\
commands will sometimes be inaccurate on camlp4-processed sources (in
particular, renaming should only be attempted for local or unexported
value bindings).

\subsection{Module packs}

Module packing is supported to the extent of its treatment in the
project description (see above), but is still experimental (and with
the limitation that ``goto'' does not go through packs while ``grep''
does, as for include directives).

\subsection{Dealing with outdated binary annotations}

\typerex\ is usually able to overcome sparse changes to the edited
files (saved or not) \emph{w.r.t} the last compiled version, and to
recompute the right positions. This feature relies on the source
snapshots which are embedded in .cmt files. This works also for
refactoring commands, but in this case a confirmation will be asked
before proceeding.

\subsection{Permissive behavior}

Some internal errors which could occur while processing some files
(for example due to unhandled language features) may be caught and
reported to the user (asking for a confirmation in the case of
refactoring). This avoids giving up too soon on errors which are
clearly harmless to a specific action.

\section{Recovery and debugging}

Except for restarting the server, this section is more intended to
developing and debugging \typerex.

\subsubsection{Errors and server restart}

If the OCP server crashes for any reason (or becomes crazy), it is
possible to restart it using
{\verbsize\begin{verbatim}
    M-x ocp-restart-server
\end{verbatim}}

\subsubsection{Logging:}

First, the \typerex\ environment for Emacs will echo minimal
information as messages in the mini-buffer, the history of
which is kept in the special buffer \verb!*Messages*!. This includes
the startup procedure, feedback about the executed commands, and in
case of unexpected error (which is a bug), a complete exception backtrace.

You may enable logging of debug information in
$\sim$\verb!/.ocp-wizard-log! by setting the \verb!ocp-debug! variable to
\verb!t! (the trace will be huge and hard to read).  The value of
\verb!ocp-debug! may also be a string, which is a comma-separated list
(without whitespace) of uncapitalized module names in the \typerex\
code.

\subsubsection{Fail fast}
In the context of debugging, it is usually easier to disable most
exception handling to get a backtrace closer to the real problem. This
can be done by setting \verb!ocp-dont-catch-errors! to \verb!t!. Note
however that this will lead \typerex\ to fail in cases which would
normally have triggered tolerant behavior.

\subsubsection{Profiling}

\typerex\ may dump profile information in
$\sim$\verb!/.ocp-wizard-profile.out! if \verb!ocp-profile! is set to
the name of a \typerex\ command (see
\verb!tools/ocp-wizard/main/owzServer.ml!). Run
\verb!profile !$\sim$\verb!/.ocp-wizard-profile.out! to generate a dot file (note
that \verb!profile! is not compiled or installed by default).
