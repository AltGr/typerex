%%%%%%%%%%%%%%%%%%%%%%%%%%%%%%%%%%%%%%%%%%%%%%%%%%%%%%%%%%%%%%%%%%%%%%%%%%
%                                                                        %
%                        TypeRex OCaml Studio                            %
%                                                                        %
%                 Thomas Gazagnaire, Fabrice Le Fessant                  %
%                                                                        %
%  Copyright 2011-2013 OCamlPro                                          %
%  All rights reserved.  This file is distributed under the terms of     %
%  the GNU Public License version 3.0.                                   %
%                                                                        %
%  TypeRex is distributed in the hope that it will be useful,            %
%  but WITHOUT ANY WARRANTY; without even the implied warranty of        %
%  MERCHANTABILITY or FITNESS FOR A PARTICULAR PURPOSE.  See the         %
%  GNU General Public License for more details.                          %
%                                                                        %
%%%%%%%%%%%%%%%%%%%%%%%%%%%%%%%%%%%%%%%%%%%%%%%%%%%%%%%%%%%%%%%%%%%%%%%%%%

\chapter{Tutorial: Building an \ocpbuild{} project}

Even if you are not using \ocpbuild{} to build you own projects, you
might need some more information to take advantage of \ocpbuild{}
features when compiling other projects.

\section{\ocpbuild{} file hierarchy}

\section{Invoking \ocpbuild{}}

\section{Setting \ocpbuild{} default parameters}

\chapter{Tutorial: How to use \ocpbuild{} in your projects}

\chapter{Building OCaml Projects with {\tt ocp-build}}

{\tt ocp-build} can be used to compile simple OCaml projects.
The tool uses simple configuration files to describe the
packages that need to be compiled, and the dependencies between
them.

Compared to other OCaml building tools, it provides the following
particularities:
\begin{itemize}
\item {\tt ocp-build} supports complete parallel builds. Its improved
  understanding of OCaml compilation constraints avoids traditionnal
  problems, arising from conflicts while compiling interfaces.
\item {\tt ocp-build} configuration files provide a simple and concise
  way to handle the complexity of OCaml projects.
\item {\tt ocp-build} supports complex compilation rules, such as
  per-file options, packing and C stubs files.
\item {\tt ocp-build} can use either a set of attributes or a digest
  of the content of a file to detect files' modifications to decide
  which files should be rebuilt.
\end{itemize}

\section{Environment Variables}

\ocpbuild{} uses the following environment variables :
\begin{description}
\item[HOME] : the user directory (``.'' if not defined)
\item[OCP\_HOME] : ocp-build configuration directory (``\$HOME/.ocp''
  if not defined)
\item[PATH] : the path of directories containing commands (separated
  by ``:'' on Unix, ``;'' on Windows)
\item[TERM] : if defined, characters are escaped (also on Windows)
\item[OCPBUILD\_VERBOSITY] : verbosity before the -v option is parsed.
\item[OCP\_DEBUG\_MODULES] : which modules to debug (need more info...)
\item[OCAMLLIB] : not directly used by \ocpbuild{}, but used by OCaml,
  from which \ocpbuild{} computes its own configuration.
\end{description}

\section{Configuration Files}

 {\tt ocp-build} uses two different kind of files to describe a project:
\begin{itemize}
\item Each package (or set of packages) should be described in a file
  with an {\tt .ocp} extension. When {\tt ocp-build} is run with the
  {\tt -scan} option, it scans the directory to find all such
  configuration files, and adds them to the project.
\item The project should be described in a file {\tt
  ocp-build.root}. This file should be at the root of the project, and
  {\tt ocp-build} will try to find it by recursively scanning all the
  parents directories. If it does not exist, it should be created using
  the {\tt -init} option.
\end{itemize}

\section{Compilation Layout}

{\tt ocp-build} generates files both in the source directories and in
a special {\tt \_obuild} directory, depending on the nature of the
files:
\begin{itemize}
\item Temporary source files and compilation garbage are stored in the
  source directories. This set includes implementation and interfaces
  files generated by {\tt ocamllex} and {\tt ocamlyacc}, and other
  special files such as {\tt .annot} files.
\item Binary object files are stored in the {\tt \_obuild} directory,
  where a sub-directory is created for each package.
\end{itemize}

\section{Format of the package description files ({\tt .ocp})}

\subsection{Description of Simple Packages}

A simple package description looks like this:

\begin{verbatim}
begin library "ocplib-system"
  files = [ "file.ml" "process.ml" ]
  requires = [ "unix" ]
end
\end{verbatim}

This description explains to {\tt ocp-build} that a library {\tt
  ocplib-system} should be built from source files {\tt file.ml} and
{\tt process.ml} (and possibly {\tt file.mli} and {\tt process.mli}),
and that this library depends on the {\tt unix} library to be built.

Another simple description is:

\begin{verbatim}
begin program "file-checker"
  files = [ "checkFiles.ml" "checkMain.ml" ]
  requires = [ "ocplib-system" ]
end
\end{verbatim}

This description tells {\tt ocp-build} that it should build an
executable {\tt file-checker} from the provided source files, and with
a dependency towards {\tt ocplib-system}. {\tt ocp-build} will
automatically add the dependency towards {\tt unix} required by {\tt
  ocplib-system}.

\subsection{OCaml Configuration}

The following variables are automatically defined by
\ocpbuild{} from OCaml configuration:
\begin{description}
\item[ocaml\_major\_version]
\item[ocaml\_minor\_version]
\item[ocaml\_point\_version]
\end{description}

\subsection{OCaml options}

\subsubsection{Per-package options}

\begin{description}
\item[dirname(string)] The directory where the package files are located.
\item[generated(bool)] If true, the package is already installed
\item[has\_byte(bool)]
\item[has\_asm(bool)]
\end{description}

\subsubsection{Per-file options}


\begin{description}
\item[ml(bool)] The file is an implementation source (.ml file)
\item[mli(bool)] The file is an interface source (.mli file)
\item[cflags(list)] Options to be passed to the C compiler
\item[ccopt(list)] ???
\item[nopervasives(bool)]
\item[nodeps(list)] A list of false dependencies
\item[nocmxdeps(list)] A list of false dependencies for native code
\item[bytelink(list)]
\item[bytecomp(list)]
\item[asmlink(list)]
\item[asmcomp(list)]
\item[dep(list)]
\item[rule\_sources(list)]
\item[pp(list)] ???
\item[pp\_requires(list)] ???
\item[sort(bool)]
\end{description}


\subsection{Advanced options}

\subsubsection{Per-file options}

Options can be specified on a per-file basis:

\begin{verbatim}
begin library "ocplib-fast"
  files = [
    "fastHashtbl.ml" (asmcomp = [ "-inline"; "30" ])
    "fastString.ml"
  ]
end
\end{verbatim}

They can also be specified for a group of files:

\begin{verbatim}
begin library "ocplib-fast"
  files = [
    begin  (asmcomp = [ "-inline"; "30" ])
    "fastHashtbl.ml"
    "fastMap.ml"
    end
    "fastString.ml"
  ]
end
\end{verbatim}


\subsubsection{Configurations}

\subsubsection{Preprocessor requirements: {\tt pp\_requires}}

The {\tt pp\_requires} option can be used to declare a dependency
between one or more source files and a preprocessor that should thus
be built before. The preprocessor must be specified as a program
package in a projet, plus the target (bytecode {\tt byte} or native
{\tt asm}):

\begin{verbatim}
begin library "ocplib-doc"
  files = [
    "docHtml.ml" (
       pp = [ "./_obuild/ocp-pp/ocp-pp.byte ]
       pp_requires = [ "ocp-pp:byte" ]
    )
    "docInfo.ml"
  ]
  requires = ["ocp-pp"]
end
\end{verbatim}

Note that you still need:
\begin{itemize}
\item To specify the package in the {\tt requires} directive, to ensure that
  this package will be available when your package will need it for processing.
\item To specify the command {\tt pp} to call the preprocessor
\end{itemize}

\section{Command line options}

\chapter{Managing Syntax Extensions with \ocpbuild{}}

\chapter{Installation}

\ocpbuild{} is part of \typerex{}. The simplest way to install it is
to use \opam{}\footnote{\url{http://opam.ocamlpro.com}}, the source package
manager for OCaml. If for some reasons, you are not satisfied by this
way, you will want to try to install it from its source repository, on
\github{}.

\section{Installing with \opam{}}

\ocpbuild{} is available in \opam{}. It is a meta-package (an empty
package) that triggers the installation of \typerex{}, with version
greater than 1.99. Indeed, \ocpbuild{} is compiled and installed by
the \typerex{} package. Any previous version of \ocpbuild{}
(especially version 0.1) should be uninstalled \underline{before}
installing \typerex{}.

First, let's check if ocp-build is already installed:
\begin{verbatim}
 peerocaml:~%  opam info ocp-build
             package: ocp-build
   installed-version: ocp-build.0.1 [4.00.1]
   available-version: 1.99.2-beta
         description: Project manager for OCaml
\end{verbatim}

The output of the command shows that \ocpbuild{} is already installed,
with version 0.1. We should remove it immediatly:

\begin{verbatim}
 peerocaml:~%  opam remove ocp-build
The following actions will be performed:
 - remove ocp-build.0.1
0 to install | 0 to reinstall | 0 to upgrade | 0 to downgrade | 1 to remove
\end{verbatim}

Note that some other packages depending on \ocpbuild{} can need to be
uninstalled too. You can keep a list of these packages, so that you
can install them again after installing the new version.

If you only ask \opam{} to install \ocpbuild{}, \opam{} might decide
to re-install \ocpbuild{} 0.1 because it has a shorter chain of
dependencies than \ocpbuild{} 1.99. To force it to install the new
version, we can ask for both \ocpbuild{} and \typerex{}:
\begin{verbatim}
 peerocaml:~%  opam install ocp-build typerex
The following actions will be performed:
 - install ocp-build.1.99.2-beta
 - install typerex.1.99.2-beta
2 to install | 0 to reinstall | 0 to upgrade | 0 to downgrade | 0 to remove
Do you want to continue ? [Y/n]
\end{verbatim}

\section{Installing from \github{}}

\ocpbuild{} sources can be retrieved from \github{}. The latest
version is developed in the {\tt typerex2} branch of the {\tt OCamlPro/typerex}
repository:

\begin{verbatim}
 peerocaml:~%  git clone git@github.com:OCamlPro/typerex.git
 peerocaml:~%  git checkout typerex2
\end{verbatim}

In the source directory ({\tt typerex}), We can now configure, compile
and install:
\begin{verbatim}
 peerocaml:~%  ./configure --prefix /usr/local/
 peerocaml:~%  make
 peerocaml:~%  make install
\end{verbatim}

The last command will install all \typerex{} commands and
libraries. If you just want to install \ocpbuild{}, you can use:

\begin{verbatim}
 peerocaml:~%  sudo ./_obuild/ocp-build/ocp-build.asm -install ocp-build \
    -install-bin /usr/local/bin -install-lib /usr/local/lib/ocaml
\end{verbatim}

Note that we used {\tt sudo} since the install paths we specified
require administrator priviledges.

It is also possible to uninstall files installed by {\tt make install}
using \ocpbuild{}:
\begin{verbatim}
 peerocaml:~%  ocp-build -uninstall typerex
\end{verbatim}

We can also use \ocpbuild{} to uninstall packages installed by
\ocpbuild{} (but it would be a bad idea to use that to uninstall
packages installed by \opam{}):
\begin{verbatim}
 peerocaml:~%  sudo ocp-build -uninstall ocp-build
\end{verbatim}


If you want to modify \ocpbuild{}, sources specific to \ocpbuild{} are
located in the {\tt tools/ocp-build} directory.
